\chapter{Supplementary figures: predicted metastable equilibrium abundances and sensitivity analyses of site-average free alkyl chains}

\begin{figure}[h]
\centering

    \begin{subfigure}[b]{\linewidth}
       	\includegraphics[width=1\linewidth]{"figs_app2/Mound OF1_thermo"}
       	\caption{MS1}
        \label{fig:MS1_thermo}
    \end{subfigure}
    \begin{subfigure}[b]{\linewidth}
    	\includegraphics[width=1\linewidth]{"figs_app2/Mound OF2_thermo"}
    	\caption{MS2}
        \label{fig:MS2_thermo}
    \end{subfigure}
    
\end{figure}

\newpage

\begin{figure}[h]\ContinuedFloat
\centering

    \begin{subfigure}[b]{\linewidth}
       	\includegraphics[width=1\linewidth]{"figs_app2/Mound OF3_thermo"}
       	\caption{MS3}
        \label{fig:MS3_thermo}
    \end{subfigure}
    \begin{subfigure}[b]{\linewidth}
    	\includegraphics[width=1\linewidth]{"figs_app2/Mound OF4_thermo"}
    	\caption{MS4}
        \label{fig:MS4_thermo}
    \end{subfigure}
    
\end{figure}

\newpage

\begin{figure}[h]\ContinuedFloat

    \begin{subfigure}[b]{\linewidth}
    	\includegraphics[width=\linewidth]{"figs_app2/Mound OF5_thermo"}
    	\caption{MS5}
        \label{fig:MS5_thermo}
    \end{subfigure}
    
    \caption{Metastable equilibrium percent abundance of sample-average free alkyl chains of Mound Spring predicted across an Eh gradient at the temperatures and bicarbonate, ammonium (or nitrate for MS5), and proton concentrations measured at (a) MS1, (b) MS2, (c) MS3, (d) MS4, and (e) MS5. The vertical dotted line indicates the lipid-predicted Eh of the sample.}
    \label{fig:MS_thermo}
\end{figure}

\begin{figure}[h]
\centering

    \begin{subfigure}[b]{\linewidth}
       	\includegraphics[width=1\linewidth]{"figs_app2/boxplot_ggplot_02bin Mound OF1 iter 999"}
       	\caption{MS1}
        \label{fig:MS1_mc}
    \end{subfigure}
    \begin{subfigure}[b]{\linewidth}
    	\includegraphics[width=1\linewidth]{"figs_app2/boxplot_ggplot_02bin Mound OF2 iter 999"}
    	\caption{MS2}
        \label{fig:MS2_mc}
    \end{subfigure}
    
\end{figure}

\newpage

\begin{figure}[h]\ContinuedFloat
\centering

    \begin{subfigure}[b]{\linewidth}
       	\includegraphics[width=1\linewidth]{"figs_app2/boxplot_ggplot_02bin Mound OF3 iter 999"}
       	\caption{MS3}
        \label{fig:MS3_mc}
    \end{subfigure}
    \begin{subfigure}[b]{\linewidth}
    	\includegraphics[width=1\linewidth]{"figs_app2/boxplot_ggplot_02bin Mound OF4 iter 999"}
    	\caption{MS4}
        \label{fig:MS4_mc}
    \end{subfigure}
    
\end{figure}

\newpage

\begin{figure}[h]\ContinuedFloat

    \begin{subfigure}[b]{\linewidth}
    	\includegraphics[width=\linewidth]{"figs_app2/boxplot_ggplot_02bin Mound OF5 iter 999"}
    	\caption{MS5}
        \label{fig:MS5_mc}
    \end{subfigure}
    
    \caption{Sensitivity analysis sample-average free alkyl chain metastable equilibrium percent abundance in Mound Spring samples (a) MS1, (b) MS2, (c) MS3, (d) MS4, and (e) MS5 after 999 iterations of Monte Carlo-style random variation of lipid HPLC peak areas by 30\% and mass spectral response factors between 0.01x and 100x. Results have a resolution of 400 calculations per iteration, binned here into 0.02 volt increments. Dark horizontal lines indicate median values for each distribution, with 50\% of the results around the median falling within the colored box (interquartile range, or IQR). Whiskers extend to an observation 1.5 times the IQR beyond this range. Values that fall outside the span of the whiskers are not indicated to reduce visual clutter.}
    \label{fig:MS_mc}
\end{figure}


\newpage

\begin{figure}[h]
\centering

    \begin{subfigure}[b]{\linewidth}
       	\includegraphics[width=1\linewidth]{"figs_app2/Empress OF1_thermo"}
       	\caption{EP1}
        \label{fig:EP1_thermo}
    \end{subfigure}
    \begin{subfigure}[b]{\linewidth}
    	\includegraphics[width=1\linewidth]{"figs_app2/Empress OF3_thermo"}
    	\caption{EP3}
        \label{fig:EP3_thermo}
    \end{subfigure}
    
\end{figure}

\newpage

\begin{figure}[h]\ContinuedFloat
\centering

    \begin{subfigure}[b]{\linewidth}
       	\includegraphics[width=1\linewidth]{"figs_app2/Empress OF4_thermo"}
       	\caption{EP4}
        \label{fig:EP4_thermo}
    \end{subfigure}
    \begin{subfigure}[b]{\linewidth}
    	\includegraphics[width=1\linewidth]{"figs_app2/Empress OF5_thermo"}
    	\caption{EP5}
        \label{fig:EP5_thermo}
    \end{subfigure}
    
    \caption{Metastable equilibrium percent abundance of sample-average free alkyl chains of Empress Pool predicted across an Eh gradient at the temperatures and bicarbonate, ammonium, and proton concentrations measured at (a) EP1, (b) EP3, (c) EP4, and (d) EP5. The vertical dotted line indicates the lipid-predicted Eh of the sample.}
    \label{fig:EP_thermo}
\end{figure}

\newpage

\begin{figure}[h]
\centering

    \begin{subfigure}[b]{\linewidth}
      	\includegraphics[width=1\linewidth]{"figs_app2/boxplot_ggplot_02bin Empress OF1 iter 999"}
      	\caption{EP1}
        \label{fig:EP1_mc}
    \end{subfigure}
    \begin{subfigure}[b]{\linewidth}
    	\includegraphics[width=1\linewidth]{"figs_app2/boxplot_ggplot_02bin Empress OF3 iter 999"}
    	\caption{EP3}
        \label{fig:EP3_mc}
    \end{subfigure}
    
\end{figure}

\newpage

\begin{figure}[h]\ContinuedFloat
\centering

    \begin{subfigure}[b]{\linewidth}
      	\includegraphics[width=1\linewidth]{"figs_app2/boxplot_ggplot_02bin Empress OF4 iter 999"}
      	\caption{EP4}
        \label{fig:EP4_mc}
    \end{subfigure}
    \begin{subfigure}[b]{\linewidth}
    	\includegraphics[width=1\linewidth]{"figs_app2/boxplot_ggplot_02bin Empress OF5 iter 999"}
    	\caption{EP5}
        \label{fig:EP5_mc}
    \end{subfigure}
    
    \caption{Sensitivity analysis sample-average free alkyl chain metastable equilibrium percent abundance in Empress Pool samples (a) EP1, (b) EP3, (c) EP4, and (d) EP5 after 999 iterations of Monte Carlo-style random variation of lipid HPLC peak areas by 30\% and mass spectral response factors between 0.01x and 100x. Results have a resolution of 400 calculations per iteration, binned here into 0.02 volt increments. Dark horizontal lines indicate median values for each distribution, with 50\% of the results around the median falling within the colored box (interquartile range, or IQR). Whiskers extend to an observation 1.5 times the IQR beyond this range. Values that fall outside the span of the whiskers are not indicated to reduce visual clutter.}
    \label{fig:EP_mc}
\end{figure}

\newpage

\begin{figure}[h]
\centering

    \begin{subfigure}[b]{\linewidth}
       	\includegraphics[width=1\linewidth]{"figs_app2/Octopus OF1_thermo"}
       	\caption{OS1}
        \label{fig:OS1_thermo}
    \end{subfigure}
    \begin{subfigure}[b]{\linewidth}
    	\includegraphics[width=1\linewidth]{"figs_app2/Octopus OF2_thermo"}
    	\caption{OS2}
        \label{fig:OS2_thermo}
    \end{subfigure}
    
    \caption{Metastable equilibrium percent abundance of sample-average free alkyl chains of Octopus Spring predicted across an Eh gradient at the temperatures and bicarbonate, ammonium, and proton concentrations measured at (a) OS1 and (b) OS2. The vertical dotted line indicates the lipid-predicted Eh of the sample.}
    \label{fig:OS_thermo}
\end{figure}

\newpage

\begin{figure}[h]
\centering

    \begin{subfigure}[b]{\linewidth}
      	\includegraphics[width=1\linewidth]{"figs_app2/boxplot_ggplot_02bin Octopus OF1 iter 999"}
      	\caption{OS1}
        \label{fig:OS1_mc}
    \end{subfigure}
    \begin{subfigure}[b]{\linewidth}
    	\includegraphics[width=1\linewidth]{"figs_app2/boxplot_ggplot_02bin Octopus OF2 iter 999"}
    	\caption{OS2}
        \label{fig:OS2_mc}
    \end{subfigure}

    
    \caption{Sensitivity analysis of sample-average free alkyl chain metastable equilibrium percent abundance in Empress Pool samples (a) OS1, and (b) OS2 after 999 iterations of Monte Carlo-style random variation of lipid HPLC peak areas by 30\% and mass spectral response factors between 0.01x and 100x. Results have a resolution of 400 calculations per iteration, binned here into 0.02 volt increments. Dark horizontal lines indicate median values for each distribution, with 50\% of the results around the median falling within the colored box (interquartile range, or IQR). Whiskers extend to an observation 1.5 times the IQR beyond this range. Values that fall outside the span of the whiskers are not indicated to reduce visual clutter.}
    \label{fig:OS_mc}
\end{figure}