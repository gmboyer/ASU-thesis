\chapter{Conclusions and Future Work}

Cell membranes have to function.

% membrane has to function. Biology has various ways of providing function in all conditions in which life exists, which likely explains why we see enormous variety in lip structures. among the possibilities of lipid structures for a given organism and set of env conditions, those that provide the best function for the least cost will likely be selected by evolution.

% in this work, we provided evidence that the oxidation state in lipids is correlated with the oxidation state of the environment, and that adaptations to provide alkyl chain functionality and membrane fluidity were oxidized in ox and reduced in red. We hypothesized that this redox directionality in ZC could be explained by the energetic advantage of making red. molecules in red. and ox. molecules in ox. Subsequent thermodynamic analysis suggested that observed distributions of lipids were stable relative to each other along a calculated redox gradient, and that the Eh that predicted maximum thermodynamic stability and potentially the best energetic cost were positively correlated with various Eh couples calculated from inorganic solutes in concurrently sampled water. The thermodynamic stability of lipid headgroups was also investigated in bacteriohopanepolyol lipids, which gave similar results to those of alkyl chains; that the Eh predicted to result in the most stable configurations of observed headgroup ratios was positively correlated with calculated values of Eh in water samples.

% Future work: be as quantitative as possible with lipid datasets to allow calculation of ZC. See if similar trends are observed with respect to concentrations of dissolved oxidized and reduced species in the surroundings or other variables. Try thermodynamic calculations of lipid stability with different sets of basis species to simulate different kinds of dominant metabolism. Using acetate concentrations rather than carbonate species might be more important for lipid predictions in primarily heterotrophic communities, etc. Constrain sources of error for better comparison of thermo predictions and analytical observation; estimation schemes to predict lipid thermodynamic properties (maybe better model exists for thermo prediction of complex lipid distribution rather than sample-averaged chains, maybe assumption of aqueous chains less useful than a liquid/crysalline phase more representative of membranes, etc.), analytical/quantitative (e.g. RFs for aminopolyols), or otherwise.







% ZC of IPL headgroups did not show carbon becoming more oxidized downstream. Is this a fluke? Artifact of the semiquantitative method for obtaining lipid abundance? If trend is real, does it translate to deviation from equilibrium in the context of hot spring redox chemistry, thereby suggesting increased biosynthetic cost? What does ZC of BHP headgroups say? How do ZC trends compare to thermodynamically-predicted relative abundances of BHP headgroups behave along a redox gradient? Is ZC predictive of behavior, or is something else going on? Addressed by thermo predictions.

