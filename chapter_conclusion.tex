\chapter{Conclusions and Future Work}

Cell membranes have to function across the entire set of conditions in which life exists, which likely explains why we see such enormous variety in possible lipid structures. Among the possibilities of lipid structures that evolution presents to an organism, it is likely that natural selection would favor the one that is the best energetic investment.

Future studies of lipid distributions should strive to be as quantitative as possible to allow estimation of weighted molecular properties such as the average oxidation state of carbon (Z\textsubscript{C}) or thermodynamic properties that would allow prediction of equilibrium abundances. Further, thermodynamic predictions of lipid abundance are most useful when details about the surrounding geochemistry are measured, such as temperature, pressure, pH, concentrations of dissolved solutes contributing to major water chemistry, possible sources of exogenous organic matter, and so on. This way, trends in abundance-weighted Z\textsubscript{C} may be tested across a variety of environmental conditions to determine if they persist outside of the four springs studied in this work.

The next steps of these thermodynamic calculations could involve the formation of lipids from a different set of basis species than the `autotrophic' set used in this work. It may be more appropriate to use acetate, for instance, for predicting lipid distributions in a predominantly heterotrophic microbial community. Attempts should be made to further constrain sources of error to facilitate comparison of thermodynamically-predicted and analytically-observed lipid abundances. Perhaps a better estimation of aminopolyol BHP abundance can be obtained by constraining sources of error introduced by lack of authentic response factors. Alternate estimation schemes could be explored to estimate lipid thermodynamic properties. Liquid-phase or crystalline-phase alkyl chains, or some combination of the two, could be included in formation reactions from aqueous basis species to simulate the formation of a membrane lipid, which may provide a better predictive model than the formation of free aqueous lipids. Alternative methods for predicting complex distributions of lipids from basis species could explored, and results could be compared to outcomes produced by the hypothetical site-averaged chains.

This work provided evidence that the abundance-weighted Z\textsubscript{C} of lipids sampled from microorganisms from four alkaline hot springs is correlated positively with the oxidation state of their environment, and that adaptations to provide alkyl chain functionality and membrane fluidity were themselves oxidized in oxidized conditions and reduced in reduced conditions. We hypothesized that this directionality of lipid adaptation was energetically cost-efficient with respect to hot spring redox chemistry. Subsequent thermodynamic analysis suggested that observed distributions of lipids were stable relative to each other along a calculated redox gradient, and that the Eh predicting maximum thermodynamic stability, and potentially the best energetic cost, were positively correlated with various Eh couples calculated from inorganic solutes in sampled hot spring water. The thermodynamic stability of lipid headgroups was also investigated in bacteriohopanepolyol lipids. This analysis gave similar results to those of alkyl chains; that the Eh predicted to result in the most stable configurations of observed headgroup ratios was positively correlated with calculated values of Eh in water samples. In sum, this work suggests that structural adaptations in lipid distributions are observed not only because they function well, because they are also cost-effective to produce within the temperature and chemical context of their surroundings.