\chapter[CONCLUSION]{Conclusion}

\section{Summary}

Lipids must function across the entire set of conditions where life exists. As such, there is an enormous variety of lipid structures in biology representing adaptation to function for their source organism, wherever they live. As was discussed in this dissertation, however, certain structural adaptations in lipids are known to function effectively in very different sets of environmental conditions; \textit{e.g.}, glycerol dialkyl glycerol tetraethers (GDGTs) can create stable lipid membranes in both thermophiles and mesophiles. Likewise, very different sets of lipid distributions may function effectively in the similar environmental conditions; \textit{e.g.}, GDGTs or non-GDGTs can create stable lipid membranes in mesophiles. Any given lipid adaptation, whether it is a double bond, a longer alkyl chain, or a particular headgroup, will have an energetic cost associated with its biosynthesis, and these costs likely change depending on the temperature, pressure, and chemical context of the surroundings.

This work provided evidence to support the hypothesis that natural selection favors convergence upon lipid compositions that maximize function while also minimizing energetic cost. First, the abundance-weighted Z\textsubscript{C} of lipids sampled from microorganisms from four alkaline hot springs was shown to correlate positively with the oxidation state of the environment, and that adaptations to provide alkyl chain functionality and membrane fluidity were themselves oxidized under oxidized conditions and reduced under reduced conditions, which I hypothesized was energetically cost-efficient. Subsequent thermodynamic analysis supported this hypothesis for intact polar lipid (IPL) alkyl chains by providing evidence that observed IPL distributions were stable relative to each other along redox gradients and correlated positively with environmental Eh. Lipid headgroups were also investigated and yielded similar results to those of alkyl chains; that hot spring Eh was correlated positively with Eh predicted by thermodynamically stable configurations of observed headgroup ratios. In sum, this work suggests structural adaptations in lipid distributions are not only present because they function for their source organisms, but because they are cost-effective to produce within the temperature and geochemical context of their surroundings.

The thermodynamic studies described in this dissertation did not include calculations involving individual reactions of biosynthesis pathways or intracellular concentrations of chemical species in their evaluation of lipid cost. Instead, net reactions to form lipids from measured concentrations of bioavailable dissolved solutes in the surrounding water alone provided evidence that observed the biosynthesis of thermophile lipid distributions may be energetically cost-effective. It is essential for cells to regulate intracellular conditions such as pH and solute concentrations for metabolic pathways to proceed. Maintaining chemical disequilibrium from the external environment requires energy. However, I would argue this energy requirement is based on how well a cell is adapted to its surroundings; energetic shortcuts to synthesize necessary biomolecules from available materials in the surroundings would offset a portion of this energy requirement to use, for instance, during cell division. This raises the possibility that conditions \textit{outside} of the cell matter more for the energetic costs of biomolecules than the conditions \textit{inside} the cell. This concept could serve as a foundation for studies in molecular biology to evaluate potential costs of biomolecules synthesized by microbial communities in a variety of natural systems and in culture.

\section{Future Work}

Studies of lipid distributions in natural systems could be mainly exploratory (\textit{i.e.}, which lipids are found under which sets of geochemical conditions) but could be supplemented by calculation of Z\textsubscript{C} or thermodynamic predictions to allow quantitative assessment of lipid distributions in terms of energetic cost. These studies could also consider the phylogeny of source organisms, which was not done here. For instance, would a cyanobacterially-dominated microbial community adapted to thrive under conditions rich in hydrogen and methane, such as those found in fluids discharged from serpentinizing systems \citep{}, tend to produce lipid compositions with reduced carbon relative to cyanobacterially-dominated communities in oxidized ocean surface water? How phylogenetically related could two microbial communities be while having drastically different lipid compositions, and how might functional necessity and energetic cost help explain these differences?

These sorts of questions could also be probed in laboratory experiments. Various growth media of known chemical compositions could be inoculated with a mixed culture to isolate sub-sets of microbes that grow best under those conditions. Thermodynamic calculations based on those described in this work might then be used to predict whether the lipid compositions of isolated sub-sets tend to be energetically favorable in the chemical conditions of their own growth media relative to all others in the experiment. Would thermodynamic predictions offer greater insight into the expression of observed lipid distributions in these various treatments than the phylogenetic compositions of the isolated sub-sets?

Future studies should attempt to be as quantitative as possible when measuring lipid abundances to allow estimation of weighted molecular properties like the average oxidation state of carbon (Z\textsubscript{C}) or partial molal thermodynamic properties such as Gibbs free energies of formation, volumes, entropies, and the like. Whether a study involves microbial communities in natural systems or in culture, the predictive power of Z\textsubscript{C} or thermodynamic calculations to describe lipid distributions may be proportional to the extent that important physical and chemical parameters are measured in the system. As such, I recommend that these studies strive to be as complete as possible when measuring temperature, pressure, and chemical composition of the natural system or growth media.

Thermodynamic calculations could also explore the formation of lipids from a different set of basis species than the `autotrophic' set used in this work. It may be more appropriate to use acetate, for instance, to predict lipid distributions in a predominantly heterotrophic microbial community. Future studies could focus on reducing sources of error to aid the comparison of thermodynamically predicted and analytically observed lipid abundances. Constraining sources of error introduced by lack of authentic response factors may offer a better estimation of aminopolyol BHP abundance. Alternate estimation schemes could be explored to estimate lipid thermodynamic properties. Liquid-phase or crystalline-phase alkyl chains, or some combination of the two, could be included in formation reactions from aqueous basis species to simulate the formation of a membrane lipid; this may offer a better predictive model than the formation of free aqueous lipids. Alternative methods for predicting complex distributions of lipids from basis species could be explored, and results could be compared to outcomes produced by the hypothetical site-averaged chains.

% extras, if needed:
% Lipid studies addressing "maximization of function". Assumed, but not tested here. Some studies have addressed function (e.g. SQ headgroups and ability to photosynthesize, or knockout membranes with no BHPs, etc.)

\section{Broader Impacts}

Studies that strive to place lipid expression into a quantitative thermodynamic framework have the potential to change how biomarkers are interpreted in the rock record.

% Concluding remarks suggested by committee:
% Potential applications: interpretation of lipid distributions in the rock record and their controls and eventually interpreting how lipid distributions are altered after deposition; Predicting which lipid biomarkers to target in astrobiology missions; New insight into the evolution of lipids in biology. Future work could compare lipid thermo calcs to (meta)genomes to show that bioenergetic trends exist regardless of organisms.

% astrobiology impact: cite paper about lipid patterns in astrobio I sent to Melissa.

Geochemistry dictates what is necessary for habitability.


%Everett wrote for NSF proposal: The development of thermodynamic data for lipid biomolecules and their biomarker degradation products will unlock the record of geochemical transformations recorded in these compounds, which will test, confirm, and challenge conclusions drawn from mineralogical, isotopic, and trace element datasets from sedimentary rocks. It is possible that biomarkers will become as useful for deciphering the geological history of sedimentary rocks as they are presently for determining the depositional environments of the original sediments
%Biomarkers hold the potential to reveal extraordinary detail about past environmental conditions that prevailed anywhere that sediments accumulate and preserve geobiochemical information. There is an impressive potential for a comprehensive understanding of global change through integrating biomarkers with many other lines of evidence from the geological record. Remarkably, biomarker research is predominantly an analytical pursuit with few experimental studies and even fewer efforts at a theoretical framework. This project will provide a foundation for that framework, which will yield novel applications in microbial physiology, evolution of biochemical pathways, climate change, and the emerging field of geobiochemistry
