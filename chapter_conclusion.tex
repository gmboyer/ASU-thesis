\chapter[CONCLUSION]{Conclusion}

Cell membranes must function across the entire set of conditions where life exists. This likely explains why there is such an enormous variety of possible lipid structures. Among the possibilities of lipids available to an organism, I hypothesized that natural selection would favor convergence upon a composition that maximizes function while also minimizing energetic cost.

% revisit three sub-hypotheses in paragraph below and briefly the result that supported them.

This work provided evidence that the abundance-weighted Z\textsubscript{C} of lipids sampled from microorganisms from four alkaline hot springs was correlated positively with the oxidation state of their environment, and that adaptations to provide alkyl chain functionality and membrane fluidity were themselves oxidized under oxidized conditions and reduced under reduced conditions. I hypothesized that this directionality of lipid adaptation was energetically cost-efficient with respect to hot spring redox chemistry. My later thermodynamic analysis suggested that observed lipid distributions were stable relative to each other along redox gradients and that the Eh predicting the best potential energetic cost were positively correlated with hot spring Eh. The relative thermodynamic stabilities of lipid headgroups were also investigated in bacteriohopanepolyol lipids and yielded similar results to those of alkyl chains; that hot spring Eh was correlated positively with Eh predicted by thermodynamically stable configurations of observed headgroup ratios. In sum, this work suggests structural adaptations in lipid distributions are not only present because they function adequately, but because they are also cost-effective to produce within the temperature and chemical context of their surroundings.

% Concluding remarks suggested by committee:
% environmental parameters dominant. Not what is inside the cell but outside.  Cell biology needs to get with the idea that geochem dictates what is necessary for habitability. Raise possibility that external environment is key to understanding the lipidome/evolution of lipids. Potential applications: interpretation of lipid distributions in the rock record and their controls and eventually interpreting how lipid distributions are altered after deposition; Predicting which lipid biomarkers to target in astrobiology missions; New insight into the evolution of lipids in biology. Future work could compare lipid thermo calcs to (meta)genomes to show that bioenergetic trends exist regardless of organisms.

Future studies of lipid distributions should strive to be as quantitative as possible to allow estimation of weighted molecular properties like the average oxidation state of carbon (Z\textsubscript{C}) or thermodynamic properties that would allow prediction of equilibrium abundances. Further, thermodynamic predictions of lipid abundance are most useful when details about the surrounding geochemistry are measured, such as temperature, pressure, pH, concentrations of dissolved solutes contributing to major water chemistry, possible sources of exogenous organic matter, and so on. This way, trends in abundance-weighted Z\textsubscript{C} may be tested across a variety of environmental conditions to determine if they persist outside the four springs studied in this work.

The next steps of these thermodynamic calculations could involve the formation of lipids from a different set of basis species than the `autotrophic' set used in this work. It may be more appropriate to use acetate, for instance, to predict lipid distributions in a predominantly heterotrophic microbial community. Future studies should further reduce sources of error to aid the comparison of thermodynamically predicted and analytically observed lipid abundances. Constraining sources of error introduced by lack of authentic response factors may offer a better estimation of aminopolyol BHP abundance. Alternate estimation schemes could be explored to estimate lipid thermodynamic properties. Liquid-phase or crystalline-phase alkyl chains, or some combination of the two, could be included in formation reactions from aqueous basis species to simulate the formation of a membrane lipid; this may offer a better predictive model than the formation of free aqueous lipids. Alternative methods for predicting complex distributions of lipids from basis species could be explored, and results could be compared to outcomes produced by the hypothetical site-averaged chains.




