\begin{abstract}
Lipids perform functions essential to life and have a variety of structures that are influenced by the organisms and environments that produced them. Lipids tend to resist degradation after cell death, leading to their widespread use as biomarkers in geobiology, though their interpretation is often tricky. Many lipid structures are shared between organisms and function in many geochemical conditions and extremes. We argue that it is more useful to interpret lipid distributions as a balance of functional necessity and energy cost. This work utilizes a quantitative thermodynamic framework for interpreting energetically driven adaptation in lipids.

Yellowstone National Park is a prime location for studying biological adaptation to a wide range of temperature and geochemical conditions. Lipids were extracted and quantified from thermophilic microbial communities sampled spatially along the temperature (29-91$^{\circ}$C) and chemical gradients of four alkaline Yellowstone hot springs. A downstream increase in the average oxidation state of carbon (Z\textsubscript{C}) was observed, caused by decreased alkyl chain carbon content, increased degree of unsaturation, and a shift from ether to ester linkage. We hypothesized that these adaptations were selected because they represent cost-effective solutions to providing thermostable membranes.

This hypothesis was explored by assessing the relative energetic favorability of autotrophic reactions to form alkyl chains from known concentrations of dissolved inorganic species at elevated temperatures. It was found that the oxidation-reduction potential (Eh) predicted to favor formation of sample-representative alkyl chains had a strong positive correlation with Eh calculated from hot spring water chemistry (R\textsuperscript{2} = 0.72). A separate thermodynamic analysis of bacteriohopanepolyol lipids found that predicted equilibrium abundances of observed polar headgroup distributions were also highly correlated with Eh of the surrounding water (R\textsuperscript{2} = 0.84). These results represent the first quantitative thermodynamic assessment of microbial lipid adaptation in natural systems, and suggest that observed lipid distributions represent energetically cost-effective assemblages along temperature and chemical gradients.
\end{abstract}