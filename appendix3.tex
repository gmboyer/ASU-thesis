\chapter{Supplementary figures: predicted equilibrium abundances of bacteriohopanepolyol polar headgroups}

\singlespace

\begin{figure}[h]
\centering
    \begin{subfigure}[b]{\linewidth}
        \includegraphics[width=\linewidth]{figs_app3/"Bison OF1_BHP_mosaic"}
        \label{fig:BP1_degform}
    \end{subfigure}
    \begin{subfigure}[b]{\linewidth}
        \includegraphics[width=\linewidth]{figs_app3/"Bison OF2_BHP_mosaic"}
        \label{fig:BP2_degform}
    \end{subfigure}
\end{figure}

\newpage

\begin{figure}[h]\ContinuedFloat
    \begin{subfigure}[b]{\linewidth}
        \includegraphics[width=\linewidth]{figs_app3/"Bison OF3_BHP_mosaic"}
        \label{fig:BP3_degform}
    \end{subfigure}\\[-4ex]
    \begin{subfigure}[b]{\linewidth}
    	\includegraphics[width=\linewidth]{figs_app3/"Bison OF4_BHP_mosaic"}
        \label{fig:BP4_degform}
    \end{subfigure}
\end{figure}

\newpage

\begin{figure}[h]\ContinuedFloat
    \begin{subfigure}[b]{\linewidth}
        \includegraphics[width=\linewidth]{figs_app3/"Bison OF5_BHP_mosaic"}
        \label{fig:BP5_degform}
    \end{subfigure}
    \begin{subfigure}[b]{\linewidth}
        \includegraphics[width=\linewidth]{figs_app3/"Bison OF6_BHP_mosaic"}
        \label{fig:BP6_degform}
    \end{subfigure}\\[-4ex]
\end{figure}

\newpage

\begin{figure}[h]\ContinuedFloat
    \begin{subfigure}[b]{\linewidth}
    	\includegraphics[width=\linewidth]{figs_app3/"Mound OF1_BHP_mosaic"}
        \label{fig:MS1_degform}
    \end{subfigure}
    \begin{subfigure}[b]{\linewidth}
        \includegraphics[width=\linewidth]{figs_app3/"Mound OF2_BHP_mosaic"}
        \label{fig:MS2_degform}
    \end{subfigure}
\end{figure}

\newpage

\begin{figure}[h]\ContinuedFloat
    \begin{subfigure}[b]{\linewidth}
        \includegraphics[width=\linewidth]{figs_app3/"Mound OF3_BHP_mosaic"}
        \label{fig:MS3_degform}
    \end{subfigure}\\[-4ex]
    \begin{subfigure}[b]{\linewidth}
    	\includegraphics[width=\linewidth]{figs_app3/"Mound OF4_BHP_mosaic"}
        \label{fig:MS4_degform}
    \end{subfigure}
\end{figure}

\newpage

\begin{figure}[h]\ContinuedFloat
    \begin{subfigure}[b]{\linewidth}
        \includegraphics[width=\linewidth]{figs_app3/"Mound OF5_BHP_mosaic"}
        \label{fig:MS5_degform}
    \end{subfigure}
    \begin{subfigure}[b]{\linewidth}
        \includegraphics[width=\linewidth]{figs_app3/"Empress OF1_BHP_mosaic"}
        \label{fig:EP1_degform}
    \end{subfigure}\\[-4ex]
\end{figure}

\newpage

\begin{figure}[h]\ContinuedFloat
    \begin{subfigure}[b]{\linewidth}
        \includegraphics[width=\linewidth]{figs_app3/"Empress OF3_BHP_mosaic"}
        \label{fig:EP3_degform}
    \end{subfigure}
    \begin{subfigure}[b]{\linewidth}
        \includegraphics[width=\linewidth]{figs_app3/"Empress OF4_BHP_mosaic"}
        \label{fig:EP4_degform}
    \end{subfigure}
\end{figure}

\newpage

\begin{figure}[h]\ContinuedFloat
    \begin{subfigure}[b]{\linewidth}
    	\includegraphics[width=\linewidth]{figs_app3/"Empress OF5_BHP_mosaic"}
        \label{fig:EP5_degform}
    \end{subfigure}\\[-4ex]
    \begin{subfigure}[b]{\linewidth}
        \includegraphics[width=\linewidth]{figs_app3/"Octopus OF1_BHP_mosaic"}
        \label{fig:OS1_degform}
    \end{subfigure}
\end{figure}

\newpage

\begin{figure}[h]\ContinuedFloat
    \begin{subfigure}[b]{\linewidth}
        \includegraphics[width=\linewidth]{figs_app3/"Octopus OF2_BHP_mosaic"}
        \label{fig:OS2_degform}
    \end{subfigure}\\[-4ex]

\caption[Predicted metastable equilibrium abundance of tetrafunctional and pentafunctional BHP headgroups in Mound Spring, Empress Pool, and Octopus Spring samples]{Predicted metastable equilibrium abundance of tetrafunctional (blue) and pentafunctional (orange) BHP headgroups in Mound Spring, Empress Pool, and Octopus Spring samples compared to observed ratios (blue to orange vertical arrow lengths) with corresponding BHP headgroup-predicted Eh indicated for each.}
\label{fig:BHP_degree_formation}
\end{figure}


% revised HKF equation
\begin{align}
\begin{split}
\Delta G^{\circ}_{aq} = & \Delta_{f}G^{\circ}_{aq} - S_{aq}^{\circ}(T - T_{r}) - c_{1}\Big[T \cdot \ln{\frac{T}{T_{r}}} - T + T_{r} \Big] \\
      - & c_{2}\Big\{\Big[\Big(\frac{1}{T - \theta}\Big) - \Big(\frac{1}{T_{r} - \theta}\Big) \Big] \Big( \frac{\theta - T}{\theta} \Big) - \frac{T}{\theta^{2}} \ln{\Big( \frac{T_{r}(T - \theta)}{T(T_{r} - \theta)} \Big)} \Big\} \\
      + & a_{1}(P - P_{r}) + a_{2}\ln{\Big(\frac{\Psi + P}{\Psi + P_{r}} \Big)} + \Big( \frac{1}{T - \theta} \Big)\Big[ a_{3}(P - P_{r}) + a_{4}\ln{\Big( \frac{\Psi + P}{\Psi + P_{r}} \Big)} \Big] \\
      + & \omega \Big( \frac{1}{\epsilon} - 1 \Big)
\end{split}
\end{align}

% apparent standard partial molal Gibbs free energy of formation of the aqueous species from the elements at the temperature and pressure of interest.
$\Delta G^{\circ}_{aq}$

% standard partial molal properties of formation of the aqueous species from the elements at 298.15K and 1 bar.
$\Delta_{f}G^{\circ}_{aq}$ = Gibbs free energy
$S_{aq}^{\circ}$ = entropy of the species

$T$ = temperature of interest
$T_{r}$ = 298.15K
