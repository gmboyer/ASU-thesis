\chapter[INTRODUCTION]{Introduction}


Lipids, or fats, are biomolecules that perform many tasks essential to life. One such task is the formation of cell membranes, providing compartmentalization of metabolic pathways that would be unable to proceed otherwise, preserving disequilibrium in chemical potential required for harvesting energy, and separating the outside of the cell from the inside. Lipids host membrane-bound proteins, serve as anchors for cell walls, perform signaling, and store energy. Because lipids fill so many biological roles, and because organisms need these roles fulfilled in so many sets of environmental conditions, the variety of known lipid structures is immense \citep[\textit{e.g.,}][]{Sturt_Intact_2004, belin2018hopanoid, van2008membrane, Yoshinaga_Systematic_2011, schouten2013organic}.

In this work, lipid function is mainly referenced in terms of the formation of lipid membranes to compartmentalize the cell or parts of the cell (\textit{e.g.,} many organelles in Eukaryotes, thylakoids in cyanobacteria, the periplasmic space in Gram-negative bacteria, \textit{etc.}). In all organisms, lipid membranes separate the inside of the cell from the outside. This barrier concentrates intracellular material to allow metabolic pathways to proceed, helps prevent harmful extracellular material from entering, and allows degrees of selectivity when importing or exporting material from the cell. Compartmentalization by lipid membranes also preserves the electrochemical gradients harnessed by organisms to harvest cellular energy, such as those produced by the electron transport chain. Compartmentalization is critically important for survival, and as such, biochemical pathways have emerged to regulate the stability of lipid membranes. The effectiveness of any given lipid membrane to compartmentalize adequately changes under different external conditions. For instance, decreasing temperature often results in a decrease in the fluidity of lipid membranes, and \textit{vice versa} for increasing temperature. If a lipid membrane becomes too rigid or too fluid, it may become leaky or rupture, disrupting the boundary between `inside' and `outside' and potentially causing great harm to the cell.

The physical properties of a lipid membrane depends on the chemical structures of the individual lipid molecules that comprise it. A simplified cartoon cross-section of a lipid membrane is shown in Figure \ref{fig:simple_membrane_bilayer}. In this cartoon, two layers are apparent at the top and bottom, which is why this particular type of lipid membrane is referred to as a `bilayer'. A typical lipid in the bilayer is also depicted schematically with its different components labelled. These components consist of a headgroup at one end, alkyl chains at the other end, and a backbone to connect them. The headgroup is typically polar and hydrophilic, or `water-loving', while the alkyl chains are nonpolar and hydrophobic, or `water-fearing'. The difference in polarity across the lipid and resulting non-covalent interaction with other polar lipid molecules is responsible for holding the lipid membrane together in an aqueous environment. Hydrophobic interactions tend to keep alkyl chains oriented toward the interior of the lipid membrane, while headgroups orient outward toward water on either side of the membrane. These lipids are often referred to in the literature as intact polar lipids, or IPLs. The word `intact' is in reference to the goal of the analytical method, which is to analyze as much of the complete lipid as possible (in contrast to analyzing non-intact lipid components, such as hydrolyzed alkyl chains). Several examples of bilayer-forming IPLs with color-coded headgroups, backbones, and alkyl chains are depicted in Figure \ref{fig:five_lipid_examples} to demonstrate that structural differences between lipids can occur in any component. For some organisms, such as many archaea, lipid membranes may be partially or exclusively comprised of lipids that form single layer, or `monolayer' (depicted in Figure \ref{fig:simple_membrane_monolayer}). Monolayer-forming lipids also have polar headgroups, backbones, and nonpolar alkyl chains. Differences in structure between components can also exist for monolayer-forming lipids, as depicted by the examples in Figure \ref{fig:five_GDGT_examples}.

A significant portion of the hydrophobic interior of cell membranes in living organisms. Many of these hydrocarbon structures are resistant to degradation and persist in the environment as recognizable pieces long after the death of the source organism. Some lipid structures are dated to billions of years old in the rock record \citep{brocks2003composition, brocks2003reconstruction}. Further, observed lipid distributions have been shown to reflect the organisms and environments that produced them. % mention archaea, or remove this sentence and combine paragraphs.

Structural diversity, longevity, and potential traceability have led to the extensive use of lipids in biogeoscience as biomarkers for predicting environmental conditions of the past or sources of organic matter. However, interpreting lipid biomarkers can be tricky due to the limited specificity of certain structures to a single organism or set of environmental variables, since many lipid structures are shared between organisms and function in a number of geochemical conditions. This is discussed in more detail in Chapter \ref{ch1}.

It is often more useful to interpret patterns in lipid distributions in the context of the fitness advantage provided to the source organisms. For instance, an abundance of \textit{cis}-double bonds in lipid chains might indicate that fitness is conferred by maintaining cell membrane fluidity in a low-temperature environment. However, the fitness advantage provided by a lipid structure, or any biomolecule for that matter, is not only determined by how well it functions, but also by its energetic cost. No matter how well a particular lipid structure might function, it must have an acceptable energetic cost to be an advantageous evolutionary investment. Furthermore, it seems likely that if several different evolutionary options are available for lipid structures that fill the same functional role with equal ability, the structure chosen by natural selection will be the one that saves the most cellular energy. Organisms synthesize lipids from the materials available to them, and it follows that an organism that makes the best use of available materials saves the most energy. Considering that functional lipids must be synthesized by organisms across the entire range of conditions inhabited by life, and that these lipids must also have an acceptable biosynthetic cost, it is no wonder that so many lipid structures and compositions are observed. 

The idea that lipid distributions are explained by natural cost-benefit analysis serves as the theoretical basis for this work. The main hypothesis of this work is that microbes adapt their lipids to \textit{maximize function} while also \textit{minimizing cost}. By `maximizing function', I refer to the process by which natural selection favors biomolecules, such as lipids, that provide a competitive advantage for their source organism by performing their biological tasks capably. This evolutionary process of maximizing lipid function is not investigated directly by this work, though it is assumed to be actively taking place in the microbial communities studied. This assumption is based primarily on prior studies that tie lipid modifications to the regulation of membrane homeostasis. For instance, increasing lipid chain length and incorporating \textit{cis}-double bonds to decrease or increase membrane fluidity and permeability, respectively \citep[see review by][]{zhang2008membrane}. The incorporation of these various lipid modifications to produce stable, functional membranes can be interpreted as a maximization of lipid function. While most of this work frames lipid function to regulation of membrane homeostasis, the enhancement of other cell processes, such as the stabilization of photosystem proteins with phosphatidylglycerol and sulfoquinovosyldiacylglycerol lipids \citep{sato2004roles}, can also be interpreted as lipid adaptation to maximize function.

Regardless of how well a lipid composition functions for an organism, it must have an acceptable biosynthetic cost. This work addresses the second half of the main hypothesis relating to the `minimization of cost'. By investigating the geochemical conditions influencing the potential cost of synthesizing lipids from available materials, I hypothesized that patterns would emerge suggesting an energetically driven evolutionary convergence on observed lipid distributions. % leads to first sub-hypothesis. Explain how it means making best use of what's available. This is addressed in Ch2 % Introduce second and third hypothesis, which are related but pertain to different portions of the lipid, and are addressed in ch3 and 4. % general approach of thermodynamic calculations showing formation of lipids from basis species. Example with phosphate and the more ambiguous kink example (2-panel fig). Rip figs from slides if necessary.

% What better place to study lipid adaptation to a wide range of temp and geochem than YNP? Introduce four study sites, with pics. Generally what is happening: downstream decrease in temp, changing chemical conditions that tend to become more oxidized, as shown in Ch2.


% This work represents the first step in placing lipid adaptation into a quantitative thermodynamic context that is predictable from temperature and chemical composition. While the work deals with primarily thermophilic microbes, I hope the bioenergetic principles of this work can be applied to non-thermophiles as well.