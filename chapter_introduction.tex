\chapter[INTRODUCTION]{Introduction}


Lipids, or fats, are biomolecules that perform many tasks essential to life. One such task is the formation of cell membranes, providing compartmentalization of metabolic pathways that would be unable to proceed otherwise, preserving disequilibrium in chemical potential required for harvesting energy, and separating the outside of the cell from the inside. Lipids host membrane-bound proteins, serve as anchors for cell walls, perform signaling, and store energy. Because lipids fill so many biological roles, and because organisms need these roles fulfilled in so many sets of environmental conditions, the variety of known lipid structures is immense \citep{Sturt_Intact_2004, belin2018hopanoid, van2008membrane, Yoshinaga_Systematic_2011, schouten2013organic}.

% in this work, mainly referring to lipids based on their role in forming membranes and compartmentalization. Outside/inside (prok and euk: outside of cell/inside of cell, Euk: organelles, cyanos: photosynthetic internal membranes/lamellae, etc.). Show simple lipid cross section. Define IPL and components, polar nature. Show example lipids with color-coded portions.

Hydrocarbons typically make up a significant portion of lipid structure. They often contribute to the hydrophobic interior of cell membranes in living organisms. Many of these hydrocarbon structures are resistant to degradation and persist in the environment as recognizable pieces long after the death of the source organism. Some lipid structures are dated to billions of years old in the rock record \citep{brocks2003composition, brocks2003reconstruction}. Further, observed lipid distributions have been shown to reflect the organisms and environments that produced them. % mention archaea, or remove this sentence and combine paragraphs.

Structural diversity, longevity, and potential traceability have led to the extensive use of lipids in biogeoscience as biomarkers for predicting environmental conditions of the past or sources of organic matter. However, interpreting lipid biomarkers can be tricky due to the limited specificity of certain structures to a single organism or set of environmental variables, since many lipid structures are shared between organisms and function in a number of geochemical conditions. This is discussed in more detail in Chapter \ref{ch1}.

It is often more useful to interpret patterns in lipid distributions in the context of the fitness advantage provided to the source organisms. For instance, an abundance of \textit{cis}-double bonds in lipid chains might indicate that fitness is conferred by maintaining cell membrane fluidity in a low-temperature environment. However, the fitness advantage provided by a lipid structure, or any biomolecule for that matter, is not only determined by how well it functions, but also by its energetic cost. No matter how well a particular lipid structure might function, it must have an acceptable energetic cost to be an advantageous evolutionary investment. Furthermore, it seems likely that if several different evolutionary options are available for lipid structures that fill the same functional role with equal ability, the structure chosen by natural selection will be the one that saves the most cellular energy. Organisms synthesize lipids from the materials available to them, and it follows that an organism that makes the best use of available materials saves the most energy. Considering that functional lipids must be synthesized by organisms across the entire range of conditions inhabited by life, and that these lipids must also have an acceptable biosynthetic cost, it is no wonder that so many lipid structures and compositions are observed. 

The idea that lipid distributions are explained by natural cost-benefit analysis serves as the theoretical basis for this work. The main hypothesis of this work is that microbes adapt their lipids to \textit{maximize function} while also \textit{minimizing cost}. By `maximizing function', I refer to the process by which natural selection favors lipids that provide a competitive advantage for their source organism by performing their biological tasks well. This evolutionary process of maximizing lipid function is not investigated directly by this work, though it is assumed to be actively taking place in the microbial communities studied. This assumption is based primarily on prior studies that tie lipid modifications to the regulation of membrane homeostasis. For instance, increasing lipid chain length and incorporating \textit{cis}-double bonds to decrease or increase membrane fluidity and permeability, respectively \citep[see review by][]{zhang2008membrane}. The incorporation of these various lipid modifications to produce stable, functional membranes can be interpreted as a maximization of lipid function. While most of this work frames lipid function to regulation of membrane homeostasis, the enhancement of other cell processes, such as the stabilization of photosystem proteins with phosphatidylglycerol and sulfoquinovosyldiacylglycerol lipids \citep{sato2004roles}, can also be interpreted as lipid adaptation to maximize function.

Regardless of how well a lipid composition functions for an organism, it must have an acceptable biosynthetic cost. This work addresses the second half of the main hypothesis regarding the `minimization of cost'. By investigating the geochemical conditions influencing the potential cost of synthesizing lipids from available materials, I hypothesized that patterns would emerge suggesting an energetically driven evolutionary convergence on observed lipid distributions. % leads to first sub-hypothesis. Explain how it means making best use of what's available. This is addressed in Ch2 % Introduce second and third hypothesis, which are related but pertain to different portions of the lipid, and are addressed in ch3 and 4. % general approach of thermodynamic calculations showing formation of lipids from basis species. Example with phosphate and the more ambiguous kink example (2-panel fig). Rip figs from slides if necessary.

% What better place to study lipid adaptation to a wide range of temp and geochem than YNP? Introduce four study sites, with pics. Generally what is happening: downstream decrease in temp, changing chemical conditions that tend to become more oxidized, as shown in Ch2.

% 





% This work represents the first step in placing lipid adaptation into a quantitative thermodynamic context that is predictable from temperature and chemical composition.