\chapter{Introduction}


Lipids, or fats, are biomolecules that perform many tasks essential to life. One such task is the formation of cell membranes that separate of the outside of the cell from the inside, providing compartmentalization of metabolic pathways that would otherwise be unable to proceed, and preserving disequilibrium in chemical potential required for harvesting energy. Lipids host membrane-bound proteins, serve as anchors for cell walls, perform signalling, and store energy. Because there are so many biological roles for lipids, and because they have been adapted to function within so many different conditions and organisms, the variety of known lipid chemical structures is immense.

Hydrocarbons typically comprise a significant portion of lipid structure, which in living organisms often contributes to the hydrophobic interior of a cell membrane. Many of these hydrocarbon structures are resistant to degradation and persist as recognizable pieces in the environment long after the death of the source organism, with some dating billions of years old in the rock record. Further, observed lipid distributions have been shown to reflect the organisms and environments that produced them. This structural diversity, longevity, and potential traceability has led to the extensive use of lipids in biogeoscience as biomarkers for predicting environmental conditions of the past or sources of organic matter. However, the interpretation of lipid biomarkers can be tricky due to the limited specificity of certain structures to a single organism or set of environmental variables, since many lipid structures are shared between organisms and function in a number of different geochemical conditions. This is discussed in more detail in Chapter \ref{ch1}.

It is often more useful to interpret patterns in lipid distributions in the context of the fitness advantage provided to the source organisms. For instance, an abundance of \textit{cis}-double bonds in lipid chains might indicate that fitness is conferred by maintaining cell membrane fluidity in a low-temperature environment. However, the fitness advantage provided by a lipid structure, or any biomolecule for that matter, is not only determined by how well it functions, but also by its energetic cost. No matter how well a particular lipid structure might function, it must have an acceptable energetic cost to be an advantageous evolutionary investment. Furthermore, it seems likely that if several different evolutionary options are available for lipid structures that fill the same functional role with approximately equal ability, the structure chosen by natural selection will be the one that saves the most cellular energy. Lipids must be synthesized out of the materials available to the source organism, and it follows that an organism that makes the best use of available materials saves the most energy. Considering that functional lipids must be synthesized in a cost-effective way by organisms across the entire range of conditions inhabited by life, it is really no wonder that so many different lipid structures and compositions are observed. 

The idea that lipid distributions can be explained or predicted based on a natural cost/benefit analysis serves as the theoretical basis upon which this work is based. We hypothesized that by investigating the geochemical conditions influencing the potential cost of synthesizing lipids from available materials, patterns would begin to emerge suggesting an energetically-driven evolutionary convergence on observed lipid distributions. This work represents the first step in placing lipid adaptation into a quantitative thermodynamic context that is predictable by temperature and chemical composition.

% could go chapter by chapter, giving 1-2 sentences about each if necessary.