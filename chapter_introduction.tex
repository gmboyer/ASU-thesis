\chapter[INTRODUCTION]{Introduction}


Lipids, or fats, are biomolecules that perform many tasks essential to life. One such task is the formation of cell membranes, providing compartmentalization of metabolic pathways that would be unable to proceed otherwise, preserving disequilibrium in chemical potential required for harvesting energy, and separating the outside of the cell from the inside. Lipids host membrane-bound proteins, serve as anchors for cell walls, perform signaling, and store energy. Because lipids fill so many biological roles, and because organisms need these roles fulfilled in so many sets of environmental conditions, the variety of known lipid structures is immense \citep[\textit{e.g.,}][]{Sturt_Intact_2004, belin2018hopanoid, van2008membrane, Yoshinaga_Systematic_2011, schouten2013organic}.

In this work, lipid function is mainly referenced in terms of the formation of lipid membranes to compartmentalize the cell or parts of the cell (\textit{e.g.,} many organelles in Eukaryotes, thylakoids in cyanobacteria, the periplasmic space in Gram-negative bacteria, \textit{etc.}). In all organisms, lipid membranes separate the inside of the cell from the outside. This barrier concentrates intracellular material to allow metabolic pathways to proceed, helps prevent harmful extracellular material from entering, and allows degrees of selectivity when importing or exporting material from the cell. Compartmentalization by lipid membranes also preserves the electrochemical gradients harnessed by organisms to harvest cellular energy, such as those produced by the electron transport chain. Compartmentalization is critically important for survival, and as such, biochemical pathways have emerged to regulate the stability of lipid membranes. The effectiveness of any given lipid membrane to compartmentalize adequately changes under different external conditions. For instance, decreasing temperature often results in a decrease in the fluidity of lipid membranes, and \textit{vice versa} for increasing temperature. If a lipid membrane becomes too rigid or too fluid, it may become leaky or rupture, disrupting the boundary between `inside' and `outside' and potentially causing great harm to the cell. % also chemical conditions - e.g. ammonia or sulfide toxicity due to permeability and then protonation, causing pH stress. Eric Boyd was on about this.

The physical properties of a lipid membrane depend on the chemical structures of the individual lipid molecules that comprise it. A simplified cartoon cross-section of a lipid membrane is depicted in Figure \ref{fig:simple_membrane_bilayer}.
\afterpage{
\singlespace
\begin{figure}[h]
\centering
\includegraphics[width=1\linewidth]{"figs_intro/simple_membrane_bilayer"}
\caption[toc caption]{simple membrane bilayer.}
\label{fig:simple_membrane_bilayer}
\end{figure}
\doublespace
\clearpage
}
In this cartoon, two layers are apparent at the top and bottom, which is why this particular type of lipid membrane is referred to as a `bilayer'. A typical lipid in the bilayer is also depicted schematically with its different components labeled. These components consist of a headgroup at one end, alkyl chains at the other end, and a backbone to connect them. The headgroup is typically polar and hydrophilic, or `water-loving', while the alkyl chains are nonpolar and hydrophobic, or `water-fearing'. The difference in polarity across the lipid and resulting non-covalent interaction with other polar lipid molecules is responsible for holding the lipid membrane together in an aqueous environment. Hydrophobic interactions tend to keep alkyl chains oriented toward the interior of the lipid membrane, while headgroups orient outward toward the water on either side of the membrane. These lipids are often referred to in the literature as intact polar lipids or IPLs. The word `intact' is in reference to the goal of the analytical method, which is to analyze as much of the complete lipid as possible (in contrast to analyzing non-intact lipid components, such as hydrolyzed alkyl chains). Several examples of bilayer-forming IPLs with color-coded headgroups, backbones, and alkyl chains are depicted in Figure \ref{fig:five_lipid_examples} to demonstrate that structural differences between lipids can occur in any component.
\afterpage{
\singlespace
\begin{figure}[h]
\centering
\includegraphics[width=1\linewidth]{"figs_intro/five_lipid_examples"}
\caption[toc caption]{five lipid examples.}
\label{fig:five_lipid_examples}
\end{figure}
\doublespace
\clearpage
}
For some organisms, such as many archaea, lipid membranes may be partially or exclusively comprised of lipids that form a single layer, or `monolayer' (depicted in Figure \ref{fig:simple_membrane_monolayer}).
\afterpage{
\singlespace
\begin{figure}[h]
\centering
\includegraphics[width=1\linewidth]{"figs_intro/simple_membrane_monolayer"}
\caption[toc caption]{simple membrane monolayer.}
\label{fig:simple_membrane_monolayer}
\end{figure}
\doublespace
\clearpage
}
Monolayer-forming IPLs also have polar headgroups, backbones, and nonpolar alkyl chains. Differences in structure between components can also exist for monolayer-forming lipids, as depicted by the examples in Figure \ref{fig:GDGT_examples}.

\afterpage{
\singlespace
\begin{figure}[h]
\centering
\includegraphics[width=0.5\linewidth]{"figs_intro/GDGT_examples"}
\caption[toc caption]{GDGT examples.}
\label{fig:GDGT_examples}
\end{figure}
\doublespace
\clearpage
}

A significant portion of the hydrophobic interior of cell membranes is comprised of aliphatic hydrocarbons (\textit{e.g.}, IPL alkyl chains). Many of these hydrocarbon structures are resistant to degradation and persist in the environment as recognizable pieces long after the death of the source organism. Some lipid structures are dated to billions of years old in the rock record \citep{brocks2003composition, brocks2003reconstruction}. Structural diversity, longevity, and potential traceability have led to the extensive use of lipids in biogeoscience as biomarkers for predicting environmental conditions of the past or sources of organic matter. However, interpreting lipid biomarkers can be tricky due to the limited specificity of certain structures to a single organism or set of environmental variables, since many lipid structures are shared between organisms and function in a number of geochemical conditions, as discussed in Chapter \ref{ch1}.

I argue it is more useful to interpret patterns in lipid distributions in the context of the fitness advantage provided to the source organisms. For instance, an abundance of \textit{cis}-double bonds in lipid chains might indicate that fitness is conferred by maintaining cell membrane fluidity in a low-temperature environment. However, the fitness advantage provided by a lipid structure, or any biomolecule for that matter, is not just determined by how well it functions, but also by its energetic cost. No matter how well a particular lipid structure might function, it must have an acceptable energetic cost to be an advantageous evolutionary investment. Furthermore, it seems likely that if several different evolutionary options are available for lipid structures that fill the same functional role with equal ability, the structure chosen by natural selection will be the one that saves the most cellular energy. Organisms synthesize lipids from the materials available to them, and it follows that an organism that makes the best use of available materials saves the most energy. Considering that functional lipids must be synthesized by organisms across the entire range of conditions inhabited by life, and that these lipids must also have an acceptable biosynthetic cost, it is no wonder that so many lipid structures and compositions are observed. 

The idea that lipid distributions are explained by natural cost-benefit analysis serves as the theoretical basis for this work. The major hypothesis of this work is that \textbf{microbes adapt their lipids to \textit{maximize function} while also \textit{minimizing cost}}. By `maximizing function', I refer to the process by which natural selection favors biomolecules, such as lipids, that provide a competitive advantage for their source organism by performing their biological tasks capably. This evolutionary process of maximizing lipid function was not investigated directly by this work, though it was assumed to be active within the sampled microbial communities. This assumption is based primarily on prior studies that tie lipid modifications to the regulation of membrane homeostasis. For instance, increasing lipid chain length and incorporating \textit{cis}-double bonds to decrease or increase membrane fluidity and permeability, respectively \citep[see review by][]{zhang2008membrane}. The incorporation of these various lipid modifications to produce stable, functional membranes can be interpreted as a maximization of lipid function. While most of this work frames lipid function to regulation of membrane homeostasis, the enhancement of other cell processes, such as the stabilization of photosystem proteins with phosphatidylglycerol and sulfoquinovosyldiacylglycerol lipids \citep{sato2004roles}, can also be interpreted as lipid adaptation to maximize function.

Regardless of how well a lipid composition functions for an organism, it must have an acceptable biosynthetic cost. This work addresses the second half of the major hypothesis that relates to the `minimization of cost'. By investigating the geochemical conditions influencing the potential cost of synthesizing lipids from available materials, I wanted to know whether patterns would emerge suggesting energetically driven evolutionary convergence on observed lipid distributions. This led to my sub-hypothesis, which is that \textbf{increasingly oxidized conditions will correspond to increasingly oxidized lipid distributions}, stemming from the idea that biosynthesis of oxidized biomolecules may be more cost-effective under oxidized conditions, and \textit{vice versa} for reduced biomolecules under reduced conditions \citep{amend1998energetics, shock2010potential, dick2011calculation, dick2013metastable}. This sub-hypothesis is addressed by the work in Chapter \ref{ch1}, where I calculate the average oxidation state of carbon in IPLs and their components sampled along temperature and chemical gradients.

The second and third sub-hypotheses of this dissertation are variations of the major hypothesis but deal with individual lipid components. The second sub-hypothesis is that \textbf{adaptations to maximize the function of lipid alkyl chains also minimize cost} and the third sub-hypothesis is that \textbf{adaptations to maximize the function of lipid headgroups also minimize cost}. The decision to separately address the energetic costs of lipid alkyl chains and headgroups was made for reasons explained in greater detail in Chapters \ref{ch2} and \ref{ch3}. The general approach used to estimate the relative energetic costs is based on modeling chemical reactions to form aqueous-phase lipid components from a chosen set of aqueous `basis species' that represent concentrations of inorganic solutes available to the primary producers of a microbial community, with the inclusion of electrons as a catch-all for the reducing power (\textit{e.g.}, the reduced form of nicotinamide adenine dinucleotide phosphate, NADPH) generated by the cell. Example reactions to form lipids from a set of basis species are shown in Figures \ref{fig:lipid_form_phospholipid} and \ref{fig:lipid_form_kink}.
\afterpage{
\singlespace
\begin{figure}[h]
\centering
\includegraphics[width=1\linewidth]{"figs_intro/lipid_form_phospholipid"}
\caption[toc caption]{lipid form phospholipid.}
\label{fig:lipid_form_phospholipid}
\end{figure}
\doublespace
\clearpage
}
\afterpage{
\singlespace
\begin{figure}[h]
\centering
\includegraphics[width=1\linewidth]{"figs_intro/lipid_form_kink"}
\caption[toc caption]{lipid form kink.}
\label{fig:lipid_form_kink}
\end{figure}
\doublespace
\clearpage
}
These represent net chemical reactions to form lipids from one possible set of basis species and intentionally do not take lipid synthesis pathways into account. Figure \ref{fig:lipid_form_phospholipid} depicts the formation of a phospholipid (top right) and an aminolipid (bottom right) from a set of basis species (left) with water (right) as a product. Note that the stoichiometry of most reactants is different for both lipids, which only differ by headgroup structure, and that each reaction generates a different number of water molecules as a byproduct. If the chemical activities (related to concentration) and thermodynamic properties of the basis species and lipids have been measured or can be estimated, and the temperature of the system has been measured, then it becomes possible to calculate thermodynamic properties of these net reactions.

It is easy to imagine that net reactions to form the phospholipid and aminolipid in Figure \ref{fig:lipid_form_phospholipid} will have different thermodynamic properties depending on temperature or availability of basis species. Under phosphate replete conditions, the phospholipid might confer a greater fitness advantage than the aminolipid. However, if the availability of phosphate diminishes, the reaction to form the phospholipid might become so energetically unfavorable that its functional benefit is no longer worth the cost. Indeed, previous studies have demonstrated that under phosphorus starvation, some organisms can switch from producing phospholipids to aminolipids, including phytoplankton \citep{van2009phytoplankton, martin2011phosphorus}, bacteria \citep{minnikin1974replacement, benning1995accumulation} and fungi \citep{riekhof2014phosphate}. For a community of microbes that do not have the genes necessary to express alternative lipids during nutrient limitation, it is easy to imagine a scenario where a dip in phosphate availability would still result in a shift in lipid composition; microbes reliant on phospholipids would be out-competed by microbes able to express lipids without phosphate. This supports the sub-hypothesis that adaptations to maximize the function of lipid headgroups also minimize cost, though this is explored further in Chapter \ref{ch3}.

Next, consider the two lipid variants shown in Figure \ref{fig:lipid_form_kink} that differ by the way a bend is introduced into an alkyl chain. This bend is produced by the incorporation of a \textit{cis}-double bond in one lipid (top right) and by a cyclopropane ring in the other (bottom right). There may be some functional advantage for an organism to incorporate a \textit{cis}-double bond rather than a cyclopropane ring, but assume for now that each modification results in an equally-capable lipid. The net reactions to form these two lipids from a set of basis species are also depicted. Note that there are differences in the stoichiometry of the reactants and products between the two lipid structures and that the thermodynamic properties of the two lipids likely differ by some amount. The properties of reactions to form these two lipids from basis species will likely be different under most sets of temperature and chemical conditions, though it is arguably not as easy to conceptualize as the example with phospholipids and phosphate starvation in Figure \ref{fig:lipid_form_phospholipid}. There may not be a great difference in energy required to produce a lipid with a double bond versus a cyclopentane ring, though I would argue that even small energetic savings might add up significantly when considering how many individual lipids an organism needs for its membranes. The thermodynamic calculations described in Chapter \ref{ch2} were performed to address how lipid alkyl chain modifications might maximize lipid function while minimizing energetic cost.

% What better place to study lipid adaptation to a wide range of temp and geochem than YNP? Introduce four study sites, with pics. Generally what is happening: downstream decrease in temp, changing chemical conditions that tend to become more oxidized, as shown in Ch2.

This work represents the first step in placing lipid adaptation into a quantitative thermodynamic context that is predictable from temperature and chemical composition. While this dissertation deals with communities of thermophilic microbes, I hope that the bioenergetic principles described herein will contribute to a better understanding of lipid distributions in a greater geochemical context.