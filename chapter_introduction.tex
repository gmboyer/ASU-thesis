\chapter{Introduction}


Lipids, or fats, are biomolecules that perform many tasks essential to life. One such task is the formation of cell membranes that separate of the outside of the cell from the inside, providing compartmentalization of metabolic pathways that would otherwise be unable to proceed, and preserving disequilibrium in chemical potential required for harvesting energy. Lipids host membrane-bound proteins, serve as anchors for cell walls, perform signalling, and store energy. Because there are so many biological roles for lipids, and because they have been adapted to function within so many different conditions and organisms, the variety of known lipid chemical structures is immense. Many of these structures, or at least recognizable pieces, are resistant to degradation after cell death and persist in the environment for great lengths of time, an some have been found in the rock record dating billions of years old.

Interpretation of lipid biomarkers is often tricky since many lipid structures are shared between organisms and function in a number of geochemical conditions.

limited exclusivity to a single organism or set of conditions.


For instance, an observed increase in the abundance of cis-double bonds among lipid chains could indicate adaptation to decreasing temperature by increasing cell membrane fluidity.


By investigating the geochemical conditions influencing the potential cost of synthesizing lipids from available materials, patterns begin to emerge suggesting energetically-driven convergence on observed lipid distributions. 


struct and lasting contribute to lipid use as biomarker for interpreting past conditions.

Interpretation of lipid biomarkers often proves to be tricky since many lipid structures are shared between organisms and function in a number of geochemical conditions. limited exclusivity 

What is useful is looking for patterns in lipid distributions that provide clues to fitness advantage. e.g. observed increase in the number of unsaturations on alkyl chains might indicate fitness imparted by increasing membrane fluidity at lower temperatures. Fitness advantage provided by a lipid structure is not only influenced by its ability to function, but also its bioenergetic cost. A particular lipid structure may work very well, but if there is a functional alternative that requires less energy to produce and saves cellular energy, the latter structure might be favored over the course of natural selection. Analyzing patterns in fitness advantage of lipid structures should then take into account the potential energetic cost, which changes based on the materials and energy sources available to an adapted organism.

This work attempts to provide a quantitative thermodynamic framework for interpreting lipid adaptation and predicting lipid distributions from geochemistry.


%abstract version:
Lipids perform functions essential to life and have a variety of structures that are influenced by the source organism and environment that produced them. Lipids tend to resist degradation after cell death, leading to their widespread use as biomarkers in geobiology, though their interpretation is often tricky because many lipid structures are shared between organisms and function in a number of geochemical conditions and extremes. We argue that it is more useful to interpret lipid structures based on functional ability and energetic cost. This work utilizes a quantitative thermodynamic framework for interpreting energetically and functionally-driven adaptation in lipid structures to temperature and geochemistry.

Yellowstone National Park (YNP) provide a prime location to study biological adaptation to a wide range of temperature and geochemical conditions. Lipids were extracted and quantified from thermophilic microbial communities sampled spatially along the temperature (20-91$^{\circ}$C) and chemical gradients of four alkaline YNP hot springs. A downstream increase in the average oxidation state of carbon (Z\textsubscript{C}) was observed, primarily caused by decreased alkyl chain carbon content, increased degree of unsaturation, and a shift from ether to ester linkage. We hypothesized that these adaptations were selected because they thermostable membranes along thermal and redox gradients.

This hypothesis was explored further by assessing the relative energetic favorability of autotrophic reactions to form alkyl chains from known concentrations of dissolved inorganic species at elevated temperatures. It was found that the oxidation-reduction potential (Eh) predicted to favor the formation of sample-representative alkyl chains correlated strongly with Eh calculated from hot spring water chemistry. A separate thermodynamic analysis performed on bacteriohopanepolyol lipids found that predicted equilibrium abundances of observed polar headgroup distributions were also highly correlated with environmental Eh. These results suggest that observed lipid alkyl chain and headgroup distributions represent energetically favorable assemblages and provide the first step toward a quantitative thermodynamic assessment of lipid adaptation in a geochemical context.




%LOOOOOOONG abstract version:
Lipids perform functions essential to life and have a variety of structures that are influenced by the source organism and environment that produced them. Lipids tend to resist degradation after cell death, leading to their widespread use as biomarkers in geobiology. Interpretation of lipid biomarkers is often tricky because many lipid structures are shared between organisms and function in a number of geochemical conditions and extremes. It is often more useful to interpret patterns in lipid distributions in the context of the fitness advantage provided to the source organisms. The fitness advantage provided by a lipid structure is not only determined by how well it functions, but also by its energetic cost. This work utilizes a quantitative thermodynamic framework for interpreting energetically and functionally-driven adaptation in lipid structures to temperature and geochemistry.

The hot springs of Yellowstone National Park (YNP) provide a prime location to study biological adaptation to a wide range of temperature and geochemical conditions. Lipids were extracted and quantified from thermophilic microbial communities sampled spatially along the temperature (20-91$^{\circ}$C) and chemical gradients of four alkaline YNP hot springs. A downstream increase in the average oxidation state of carbon (Z\textsubscript{C}) was observed in each spring, indicating that carbon in microbial lipids was most oxidized in low-temperature, oxidized samples and most reduced in high-temperature, reduced samples. These downstream trends in Z\textsubscript{C} were mainly caused by modification of alkyl chains to provide thermostable membranes, such as decreasing numbers of aliphatic carbons, increasing degree of unsaturation, and shifting from ether to ester chain linkage. Because oxidized and reduced alkyl chain adaptations were observed in oxidized and reduced conditions, respectively, we hypothesized that these structures were selected because they offer cost-effective solutions for providing thermostable membranes along thermal and redox gradients.

This hypothesis was explored further by assessing the relative energetic favorability of autotrophic reactions to form alkyl chains from known concentrations of dissolved inorganic species at elevated temperatures. It was found that the oxidation-reduction potential (Eh) predicted to favor the formation of sample-representative alkyl chains correlated strongly with Eh calculated from hot spring water chemistry. A separate thermodynamic analysis performed on bacteriohopanepolyol lipids found that predicted equilibrium abundances of observed polar headgroup distributions were also highly correlated with environmental Eh. These results suggest that observed lipid alkyl chain and headgroup distributions represent energetically favorable assemblages and provide the first step toward a quantitative thermodynamic assessment of lipid adaptation in a geochemical context.